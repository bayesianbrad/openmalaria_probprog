\documentclass[]{scrartcl}

%opening
\usepackage[round]{natbib}
\usepackage{listings}
% \usepackage{minted}
\usepackage{color}
\usepackage{algorithm}
\usepackage{algpseudocode}
\usepackage{amsmath}
\usepackage{amssymb}
\usepackage{dsfont}
\usepackage{pgf}
\usepackage{subcaption}
\usepackage{tikz}
\usepackage{hyperref}
\usepackage{dsfont}
\usetikzlibrary{bayesnet}
\newcommand{\iden}{\ensuremath{\mathbb{I}}}  %identity func
\newcommand{\real}{\ensuremath{\mathbb{R}}} %real numbers
\newcommand{\nat}{\ensuremath{\mathbb{N}}}  %natural numbers
\newcommand{\ints}{\ensuremath{\mathbb{Z}}}
\newcommand{\expt}{\ensuremath{\mathbb{E}}}
\newcommand{\X}{\ensuremath{\mathbb{X}}}
\newcommand{\Z}{\ensuremath{\mathfrak{Z}}}
\newcommand{\y}{\textbf{y}} 
\newcommand{\x}{\textbf{x}} 
\newcommand{\asto}{\overset{a.s.}{\to}}

\newcommand{\om}{\textit{OpenMalaria}}
\DeclareMathOperator*{\argmax}{arg\,\!max}

\title{Notes on Open Malaria}
\author{Bradley Gram-Hansen}

\begin{document}

\maketitle

\begin{abstract}

\end{abstract}

% \input{mognotes/mixofgauss}


\section{Simulation Options}

The \om simulator is designed for simulating individual scenarios. 
In order to simulate a study covering variations in several factors, one 
would have to  design a fully \textit{factorial} experiment.

\begin{quote}
	\begin{itemize}
		\item \textbf{\textit{Sweep}} : A sweep is one of more covarying factors, and set of all \textit{arms} of this sweep.
		\item \textbf{\textit{Arm}}: Each sweep must have one or more values for each of its factors. An \textit{arm} 
		is one of these combinations of values - a configuration assigning a value to each factor on the sweep. 
		\item \textbf{\textit{Experiment}}: Largely synonymous with a study, an \textit{experiment} is the complete
		description of all sweeps and arms used, together with the scenarios generated and results produced. Each scenario is generated by choosing
		one arm for each sweep in the experiment. 
		\item \textbf{\textit{Full factorial design}} : A \textit{full factorial experiment} is one where all possible scenarios are generated: all
		combinations of one arm per sweep are used.
		\item \textbf{\textit{Seed}}: Each time a scenario is simulated, the random number generated must be started from some \textit{seed}.
	\end{itemize}
\end{quote}

\end{document}

Since OpenMalaria simulations are stochastic, they use a number of random seeds (say 2-50) 
to provide estimates of the contributions of random variation to the results.